% -*-coding: utf-8 -*-

\appendix
\setcounter{chapter}{0} \setcounter{section}{0}
\setcounter{figure}{0} \setcounter{table}{0}
\setcounter{equation}{0}
\renewcommand{\thechapter}{\Alph{chapter}}

\renewcommand{\fpath}{./appendix/figures/}
%%%%%%%%%%%%%%%%%%%%%%%%%%%%%%%%%%%%%%%%%%%%%%%%%%%%%%%%%
\BiAppChapter{辐射流体力学方程组的~``Manufactured" 解}{Full Appendix}\label{FuluA}
为了确定算法的阶,需要一种计算该算法截断误差的方法。因为这需要获得该系统的精确解或者非常接近的估计解,但对于辐射流体方程组来说,这是具有困难的。为了克服这一难点,本文采用~MMS(Method of Manufactured Solutions) 方法~\cite{2017Second}。

%%%%%%%%%%%%%%%%%%%%%%%%%%%%%%%%%%%%%%%%%%%%%%%%%%%%%%%%%
\BiSection{MMS 算法得到模型的解}{Section in Appendix}
假定精确解为函数形式,并使用该函数形式推导出强迫函数组,而这些强迫函数组可用于通过我们的设计的数值方法来再现这些解。

假设精确流体解的函数形式为
\begin{equation}\label{exact_solution}
\left\{
\begin{array}{lll}
\rho=(\sin(x-t)+2),\vspace{1ex}\\
u=(\cos(x-t)+2),\vspace{1ex}\\
p=\alpha*(\cos(x-t)+2).\vspace{1ex}\\
\end{array}
\right.
\end{equation}
利用状态方程~(\ref{prefect_gas_EOS}),可得
\begin{equation*}\label{exact_solution_T}
T=\frac{\alpha\gamma(\cos(x-t)+2)}{\sin(x-t)+2}.
\end{equation*}
因此,有以下关系式,
\begin{equation*}\label{exact_solution_relationship1}
\left\{
\begin{array}{lllll}
\frac{\partial\rho}{\partial t}=-\cos(x-t),\vspace{1ex}\\
\frac{\partial\rho}{\partial x}=\cos(x-t),\vspace{1ex}\\
\frac{\partial u}{\partial t}=\sin(x-t),\vspace{1ex}\\
\frac{\partial u}{\partial x}=-\sin(x-t),\vspace{1ex}\\
\frac{\partial p}{\partial x}=-\alpha \sin(x-t),\vspace{1ex}\\
\end{array}
\right.
\end{equation*}

\begin{equation*}\label{exact_solution_relationship2}
\left\{
\begin{array}{lllll}
\frac{\partial T}{\partial t}=\frac{\alpha\gamma+2\alpha\gamma(\sin(x-t)+\cos(x-t))}{(\sin(x-t)+2)^2},\vspace{1ex}\\
\frac{\partial T}{\partial x}=\frac{-\alpha\gamma-2\alpha\gamma(\sin(x-t)+\cos(x-t))}{(\sin(x-t)+2)^2},\vspace{1ex}\\
\frac{\partial^2 T}{\partial x^2}=\frac{2\alpha\gamma}{(\sin(x-t)+2)^3}[1+\sin(x-t)\cos(x-t)+2\sin(x-t)-\cos(x-t)],\vspace{1ex}\\
\frac{\partial E}{\partial t}=\frac{\alpha}{\gamma-1}\frac{1+2(\sin(x-t)+\cos(x-t))}{(\sin(x-t)+2)^2}+(\cos(x-t)+2)\sin(x-t),\vspace{1ex}\\
\frac{\partial E}{\partial x}=-\frac{\alpha}{\gamma-1}\frac{1+2(\sin(x-t)+\cos(x-t))}{(\sin(x-t)+2)^2}-(\cos(x-t)+2)\sin(x-t),\vspace{1ex}\\
\end{array}
\right.
\end{equation*}

将以上关系式代入系统~(\ref{equilibrium_model}) 可得
\begin{equation}\label{equilibrium_model_MMS}
\left\{
\begin{array}{lll}
 \partial_t&\rho+\partial_x(\rho {u})=Q_\rho,\vspace{1ex}\\
 \partial_t&(\rho {u})+\partial_x(\rho {u}^2+p^*)=Q_u,\vspace{1ex}\\
 \partial_t&(\rho E^*)+\partial_x(( E^*+p^*){u})=Q_E,
\end{array}
\right.
\end{equation}
其中,
\begin{equation}\label{equilibrium_model_Q_rho}
Q_\rho=\cos2(x-t)+\cos(x-t)-2\sin(x-t),
\end{equation}
\begin{equation}\label{equilibrium_model_Q_u}
\begin{array}{rl}
Q_u&=-\cos(x-t)[\cos(x-t)+2]+[\sin(x-t)+2]\sin(x-t)+\cos(x-t)[\cos(x-t)+2]^2\vspace{1ex}\\
&+2[\sin(x-t)+2][\cos(x-t)+2](-\sin(x-t))-\alpha \sin(x-t)\vspace{1ex}\\
&+\frac{4}{3}\mathcal{P}_0\frac{\alpha^3\gamma^3(\cos(x-t)+2)^3}{(\sin(x-t)+2)^3}\frac{-\alpha\gamma-2\alpha\gamma(\sin(x-t)+\cos(x-t))}{(\sin(x-t)+2)^2},
\end{array}
\end{equation}
\begin{equation}\label{equilibrium_model_Q_E}
\begin{array}{rl}
Q_E&=-\cos(x-t)[\frac{\alpha(\cos(x-t)+2)}{(\gamma-1)((\sin(x-t)+2)}+\frac{1}{2}(\cos(x-t)+2)^2]\vspace{1ex}\\
&+(\sin(x-t)+2)[\frac{\alpha}{\gamma-1}\frac{1+2(\sin(x-t)+\cos(x-t))}{(\sin(x-t)+2)^2}+(\cos(x-t)+2)\sin(x-t)]\vspace{1ex}\\
&+4\mathcal{P}_0\frac{\alpha^3\gamma^3(\cos(x-t)+2)^3}{(\sin(x-t)+2)^3}\frac{\alpha\gamma+2\alpha\gamma(\sin(x-t)+\cos(x-t))}{(\sin(x-t)+2)^2}\vspace{1ex}\\
&-\sin(x-t)[(\sin(x-t)+2)(\frac{\alpha(\cos(x-t)+2)}{(\gamma-1)(\sin(x-t)+2)}+\frac{1}{2}(\cos(x-t)+2)^2)+\alpha(\cos(x-t)+2)\vspace{1ex}\\
&+\frac{4}{3}\mathcal{P}_0\frac{\alpha^4\gamma^4(\cos(x-t)+2)^4}{(\sin(x-t)+2)^4}]\vspace{1ex}\\
&+[\cos(x-t)+2][\cos(x-t)(\frac{\alpha(\cos(x-t)+2)}{(\gamma-1)(\sin(x-t)+2)}+\frac{1}{2}(\cos(x-t)+2)^2)-\alpha \sin(x-t)\vspace{1ex}\\
&+(\sin(x-t)+2)[-\frac{\alpha}{\gamma-1}\frac{1+2(\sin(x-t)+\cos(x-t))}{(\sin(x-t)+2)^2}-(\cos(x-t)+2)\sin(x-t)]\vspace{1ex}\\
&+\frac{16}{3}\mathcal{P}_0\frac{\alpha^3\gamma^3(\cos(x-t)+2)^3}{(\sin(x-t)+2)^3}\frac{-\alpha\gamma-2\alpha\gamma(\sin(x-t)+\cos(x-t))}{(\sin(x-t)+2)^2}]\vspace{1ex}\\
&-4\mathcal{P}_0(3\frac{\alpha^2\gamma^2(\cos(x-t)+2)^2}{(\sin(x-t)+2)^2})\frac{\alpha\gamma+2\alpha\gamma(\sin(x-t)+\cos(x-t))^2}{(\sin(x-t)+2)^4}\vspace{1ex}\\
&-\frac{\alpha^4\gamma^4(\cos(x-t)+2)^3}{(\sin(x-t)+2)^3}\frac{2(\cos(x-t)-\sin(x-t))(\sin(x-t)+2)^2-2(\sin(x-t)+2)\cos(x-t)[1+2(\sin(x-t)+\cos(x-t))]}{(\sin(x-t)+2)^4}.\vspace{1ex}
\end{array}
\end{equation}
使用函数~(\ref{equilibrium_model_Q_rho}), (\ref{equilibrium_model_Q_u}), (\ref{equilibrium_model_Q_E}) 作为~(\ref{equilibrium_model_MMS}) 的源项,方程~(\ref{exact_solution}) 便是系统~(\ref{equilibrium_model_MMS}) 的精确解。可以将这些函数用来计算时间和空间上辐射流体力学算法的精度。
